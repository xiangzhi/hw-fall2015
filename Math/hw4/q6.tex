%question 6

As we are finding the maximum area with a given parameter,$p$, we know that the objective function must be
\begin{equation*}
f(x,y) = xy
\end{equation*}
with the constraint of 
\begin{equation*}
2y + 2x - p = 0
\end{equation*}
First we construct the Lagrangian
\begin{equation*}
L(x,y,\alpha) = xy + \alpha(2y + 2x - p)
\end{equation*}
We then compute the gradient at set it to 0
\begin{equation*}
\Delta L(x,y,\alpha) = 
\begin{pmatrix}
y + 2\alpha\\
x + 2\alpha\\
2x + 2y -p \\
\end{pmatrix}
= \vec{0}
\end{equation*}
We are now given 3 equations
\begin{equation}\label{e1}
y = 2\alpha
\end{equation}
\begin{equation}\label{e2}
x = 2\alpha
\end{equation}
\begin{equation}
\begin{aligned}\label{e3}
2x &= -2y + p \\
x = \frac{-2y + p}{2}
\end{aligned}
\end{equation}
Insert \ref{e3} into \ref{e2}, we get
\begin{equation}\label{e4}
2 \alpha = \frac{-2y + p}{2}
\end{equation}
Then, insert \ref{e4} into \ref{e1}, we can solve $y$
\begin{equation}\label{e5}
\begin{aligned}
y &= \frac{-2y + p}{2}\\
2y &= -2y + p\\
y = \frac{p}{4}
\end{aligned}
\end{equation}
We then insert \ref{e5} back into \ref{e2} and we solve $x$
\begin{equation}
x = \frac{p}{4}
\end{equation}
We have now found the critical points in the equation, where $x = \frac{p}{4}$ and $y=\frac{p}{4}$. Notice that $x = y$. To show that we are achieving the maximum, we need to verify the second order sufficiency where we need to show the following matrix
\begin{equation}
L(x^*) = \Delta^2f(x*) + \alpha^T\Delta^2h(x^*)
\end{equation}
is negative semi-definate on $m$ which is $m = {y | \Delta h (x^*) y = 0}$. First we solve the partial derivatives for $f(x)$ and $h(x)$
\begin{equation*}
\begin{aligned}
\frac{\partial}{\partial^2 x} &= 0\\
\frac{\partial}{\partial^2 y} &= 0\\
\frac{\partial}{\partial xy} &= 1\\
\frac{\partial}{\partial yx} &= 1\\
\Delta^2 h &= 0\\
\end{aligned}
\end{equation*}
We have solved $L(x^*)$
\begin{equation*}
L(x^*) = 
\begin{pmatrix}
0 &1\\
1 &0\\
\end{pmatrix}
\end{equation*}
We then check where the matrix is positive semi-definite or negative semi-definite by choosing a vector from the subspace $M$ and apply $y^TLy$.
\begin{equation}
\begin{aligned}
\Delta h(x) &= [2 2]\\
[2 2]y &= 0\\
\begin{pmatrix}
y_1\\
y_2 \\
\end{pmatrix}
& = 
\begin{pmatrix}
-1 \\
1 \\
\end{pmatrix}
\end{aligned}
\end{equation} 
\begin{equation}
\begin{pmatrix}
-1 & 1 
\end{pmatrix}
\begin{pmatrix}
0 & 1 \\
1 & 0\\
\end{pmatrix}
\begin{pmatrix}
-1 \\
1
\end{pmatrix}
= -2
\end{equation}
This shows the matrix is negative semi-definite and therefore, $x^*$ must be the maximum of $f$.
