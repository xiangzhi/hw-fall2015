\documentclass{article}
\usepackage{graphicx}
\usepackage{amsmath}
\usepackage[framed,numbered]{matlab-prettifier}

\begin{document}

\title{Math 16-811 - HW1}
\author{Xiang Zhi Tan}

\maketitle

\section{Question 1}
implemented code:
\lstinputlisting[style=Matlab-editor]{lduSolver.m}
the code is a function that takes in A and returns the value [P,L,D,U]
\section{Question 2}
\subsection*{2.a}
Let $ L = I, A' = A$ \\
\begin{tabular}{|l|c|p{5cm}|}
\hline
\textbf{L} & \textbf{A\'} & \textbf{Operations}\\
\hline
$\begin{pmatrix}
    1 & 0 & 0 \\
    \frac{1}{2} & 1 & 0 \\
    \frac{1}{4} & 0 & 1 \\
\end{pmatrix}$
&
$\begin{pmatrix}
    4 & 7 & 0 \\
    0 & -\frac{3}{2} & -6 \\
    0 & \frac{1}{4} & 1 \\
\end{pmatrix}$
&
$R_2 = R_2 - \frac{1}{2} R_1 $ \newline
$R_3 = R_3 - \frac{1}{4} R_1$ \\
\hline
$\begin{pmatrix}
    1 & 0 & 0 \\
    \frac{1}{2} & 1 & 0 \\
    \frac{1}{4} & -\frac{1}{6} & 1 \\
\end{pmatrix}$
&
$\begin{pmatrix}
    4 & 7 & 0 \\
    0 & -\frac{3}{2} & -6 \\
    0 & 0 & 0 \\
\end{pmatrix}$
&
$R_3 = R_3 + \frac{1}{6} R_1$ \\
\hline
\end{tabular} \\
Therefore: \\
$P =
\begin{pmatrix}
    1 & 0 & 0 \\
    0 & 1 & 0 \\
    0 & 0 & 1 \\
\end{pmatrix}
$ \\
$L = 
\begin{pmatrix}
    1 & 0 & 0 \\
    \frac{1}{2} & 1 & 0 \\
    \frac{1}{4} & -\frac{1}{6} & 1 \\
\end{pmatrix}$\\
$D = 
\begin{pmatrix}
    4 & 0 & 0 \\
    0 & -\frac{3}{2} & 0 \\
    0 & 0 & 0 \\
\end{pmatrix}
$ \\
$
U = 
\begin{pmatrix}
    1 & \frac{7}{4} & 0 \\
    0 & 1 & 4 \\
    0 & 0 & 1 \\
\end{pmatrix}
$ \\
SVD Decomposition:\\
$
u =
\begin{pmatrix}
   -0.8478  &  0.4286 &  -0.3123 \\
   -0.4908  & -0.8571 &   0.1562 \\
   -0.2008  &  0.2857 &   0.9370 \\
\end{pmatrix}$\\

$s =
\begin{pmatrix}
    9.0554    &     0     &    0\\
         0   & 5.7446    &     0\\
         0     &    0  &  0.0000\\

\end{pmatrix}$\\

$v =
\begin{pmatrix}
   -0.5051   & 0.0497  &  0.8616 \\
   -0.8081   & 0.3233  & -0.4924 \\
    0.3030   & 0.9450  &  0.1231 \\
\end{pmatrix}$

\subsection*{2.b}
Let $ L = I, A' = A$ \\
\begin{tabular}{|l|c|p{5cm}|}
\hline
\textbf{L} & \textbf{A\'} & \textbf{Operations}\\
\hline
$\begin{pmatrix}
    1 & 0 & 0 & 0 \\
    \frac{1}{2} & 1 & 0 & 0 \\
    0 & 0 & 1 & 0 \\
    0 & 0 & 0 & 1 \\
    0 & 0 & 0 & 0 \\
\end{pmatrix}$
&
$\begin{pmatrix}
    4 & 8 & 0 & 0 \\
    0 & -4 & -2 & 0 \\
    0 & 4 & -1 & 0 \\
    0 & -2 & 0 & 2 \\
    0 & 0 & 2 & -1 \\
\end{pmatrix}$
&
$R_2 = R_2 - \frac{1}{2} R_1 $ \\
\hline
$\begin{pmatrix}
    1 & 0 & 0 & 0 \\
    \frac{1}{2} & 1 & 0 & 0 \\
    0 & 1 & 1 & 0 \\
    0 & -\frac{1}{2} & 0 & 1 \\
    0 & 0 & 0 & 0 \\
\end{pmatrix}$
&
$\begin{pmatrix}
    4 & 8 & 0 & 0 \\
    0 & -4 & -2 & 0 \\
    0 & 0 & -3 & 0 \\
    0 & 0 & 0 & 2 \\
    0 & 0 & 2 & -1 \\
\end{pmatrix}$
&
$R_3 = R_3 + R_2$ \newline
$R_4 = R_3 - \frac{1}{2} R_2$\\
\hline
$\begin{pmatrix}
    1 & 0 & 0 & 0 \\
    \frac{1}{2} & 1 & 0 & 0 \\
    0 & 1 & 1 & 0 \\
    0 & -\frac{1}{2} & 0 & 1 \\
    0 & 0 & \frac{2}{3} & \frac{1}{2} \\
\end{pmatrix}$
&
$\begin{pmatrix}
    4 & 8 & 0 & 0 \\
    0 & -4 & -2 & 0 \\
    0 & 0 & -3 & 0 \\
    0 & 0 & 0 & 2 \\
    0 & 0 & 0 & 0 \\
\end{pmatrix}$
&
$R_5 = R_5 + \frac{2}{3}R_3$ \newline
$R_5 = R_5 + \frac{1}{2}R_4$\\
\hline
\end{tabular} \\
Therefore: \\
$P =
\begin{pmatrix}
    1 & 0 & 0 & 0 & 0\\
    0 & 1 & 0 & 0 & 0\\
    0 & 0 & 1 & 0 & 0\\
    0 & 0 & 0 & 1 & 0\\
    0 & 0 & 0 & 0 & 1\\
\end{pmatrix}$ \\
$L = 
\begin{pmatrix}
    1 & 0 & 0 & 0 \\
    \frac{1}{2} & 1 & 0 & 0 \\
    0 & 1 & 1 & 0 \\
    0 & -\frac{1}{2} & 0 & 1 \\
    0 & 0 & \frac{2}{3} & \frac{1}{2} \\
\end{pmatrix}$ \\
$D = 
\begin{pmatrix}
    4 & 0 & 0 & 0 \\
    0 & -4 & 0 & 0 \\
    0 & 0 & -3 & 0 \\
    0 & 0 & 0 & 2 \\
    0 & 0 & 0 & 0 \\
\end{pmatrix}$ \\
$U = 
\begin{pmatrix}
    1 & 2 & 0 & 0 \\
    0 & 1 & \frac{1}{2} & 0 \\
    0 & 0 & 1 & 0 \\
    0 & 0 & 0 & 1 \\
    0 & 0 & 0 & 0 \\
\end{pmatrix}$ \\
SVD Decomposition \\
$u = 
\begin{pmatrix}
   -0.9005  & -0.0155  &  0.3562 &  -0.1617 &  -0.1895 \\
   -0.0891  & -0.7311  &  0.1847  &  0.5290 &   0.3789 \\
   -0.3787  &  0.0340  & -0.7352 &  -0.1479  &  0.5413 \\
    0.1942  & -0.3642  &  0.2836 &  -0.8023  &  0.3248 \\
    0.0078  &  0.5757  &  0.4670  &  0.1686  & 0.6496 \\
\end{pmatrix}$

$s =
\begin{pmatrix}
    9.8844    &     0   &      0     &    0\\
         0 &  3.3506   &      0     &    0\\
         0    &     0  &  2.3134    &     0\\
         0    &     0    &     0   & 1.9288\\
         0    &     0     &    0    &     0\\
\end{pmatrix}$

$v =
\begin{pmatrix}
   -0.3824  & -0.4549  &  0.7755  &  0.2132 \\
   -0.9214  &  0.2210  & -0.2847  & -0.1456 \\
    0.0579  &  0.7699  &  0.5619  & -0.2970 \\
    0.0385  & -0.3892  &  0.0434  & -0.9193 \\
 \end{pmatrix}$

\subsection*{2.c}
Let $ L = I, A' = A$ \\
\begin{tabular}{|l|c|p{5cm}|}
\hline
\textbf{L} & \textbf{A\'} & \textbf{Operations}\\
\hline
$\begin{pmatrix}
    1 & 0 & 0 \\
    \frac{3}{2} & 1 & 0 \\
    \frac{1}{2} & 0 & 1 \\
\end{pmatrix}$
&
$\begin{pmatrix}
    2 & 2 & 5 \\
    0 & -1 & -\frac{5}{2} \\
    0 & 0 & \frac{5}{2} \\
\end{pmatrix}$
&
$R_2 = R_2 - \frac{3}{2} R_1 $ \newline
$R_3 = R_3 - \frac{1}{2} R_1$ \\
\hline
\end{tabular} \\
Therefore: \\
$P =
\begin{pmatrix}
    1 & 0 & 0 \\
    0 & 1 & 0 \\
    0 & 0 & 1 \\
\end{pmatrix}
$ \\
$L = 
\begin{pmatrix}
    1 & 0 & 0 \\
    \frac{3}{2} & 1 & 0 \\
    \frac{1}{2} & 0 & 1 \\
\end{pmatrix}$\\
$D = 
\begin{pmatrix}
    2 & 0 & 0 \\
    0 & -1 & 0 \\
    0 & 0 & \frac{5}{2} \\
\end{pmatrix}
$ \\
$
U = 
\begin{pmatrix}
    1 & 1 & \frac{5}{2} \\
    0 & 1 & \frac{5}{2} \\
    0 & 0 & 1 \\
\end{pmatrix}
$ \\
SVD Decomposition\\
$
u =
\begin{pmatrix}
   -0.5859  & -0.0444 &  -0.8091 \\
   -0.6231  & -0.6138 &   0.4849 \\
   -0.5182  &  0.7882 &   0.3319 \\
\end{pmatrix}$\\

$s =
\begin{pmatrix}
    9.7910   &      0    &     0\\
         0   & 1.4162    &     0\\
         0   &      0   & 0.3606\\

\end{pmatrix}$\\

$v =
\begin{pmatrix}
   -0.3635 &  -0.8063  &  0.4666\\
   -0.2999 &  -0.3729 &  -0.8781\\
   -0.8820  &  0.4591 &   0.1062\\
\end{pmatrix}$
\section{Question 3}
\subsection*{3.a}
Because the determinent of A is 5, this means A is invertible, therefore $\bar{x}$ solution is the only solution. \\
SVD Decomposition\\
$u =
\begin{pmatrix}
   -0.3635 &   0.8063 &   0.4666 \\
   -0.2999 &   0.3729 &  -0.8781\\
   -0.8820 &  -0.4591 &   0.1062\\
\end{pmatrix}$\\

$s =
\begin{pmatrix}
    9.7910   &      0      &   0\\
         0  &  1.4162      &   0\\
         0   &      0  &  0.3606\\
\end{pmatrix}$\\

$v =
\begin{pmatrix}
   -0.5859 &   0.0444  & -0.8091 \\
   -0.5182 &  -0.7882  &  0.3319 \\
   -0.6231 &   0.6138  &  0.4849 \\
\end{pmatrix}$
\begin{equation*}
\bar{x} = V\frac{1}{\Sigma}U^Tb = 
\begin{pmatrix}
-12.3909 \\
10.2365 \\
9.5748
\end{pmatrix}
\end{equation*}

\subsection*{3.b}
Since the determinent of A is 0. This means the system either has no solution or that $\bar{x}$ is the closet solution depending whether $b \in cols(A)$. However, since $b \not{\in} cols(A)$. This mean there is no solution for the system.\\
SVD Decomposition\\
$u =
\begin{pmatrix}
   -0.8478  &  0.4286  & -0.3123 \\
   -0.4908  & -0.8571  &  0.1562 \\
   -0.2008 &   0.2857  &  0.9370 \\
\end{pmatrix}$\\

$s =
\begin{pmatrix}
    9.0554     &    0    &     0\\
         0   & 5.7446    &     0\\
         0   &      0   & 0.0000\\
\end{pmatrix}$\\

$v =
\begin{pmatrix}
   -0.5051  &  0.0497  &  0.8616\\
   -0.8081  &  0.3233  & -0.4924\\
    0.3030  &  0.9450  &  0.1231\\
\end{pmatrix}$
\subsection*{3.c}
Since the determinent of A is 0. This means the system either has no solution or that $\bar{x}$is the closet solution depending whether $\bar{x} \in cols(A)$. To show that $\bar{x} \in cols(A$ I used the gaussian elimination method. \\
\begin{tabular}{l|p{7cm}}
\hline
$\left(
\begin{array}{rrr|r}
   4 &  7  &  0 & 18\\
   8 &  14 &  0 & 36\\
   1  & 2 &  1 & 8\\
\end{array}
\right) $
&
$R_2 = R_2 + 6R_1$ \\
\hline
$\left (
\begin{array}{rrr|r}
   4 &  7  &  0 & 18\\
   0 &  0 &  0 & 0\\
   1  & 2 &  1 & 8\\
\end{array}  \right )$
&
$R2 = \frac{R2}{2}$ \newline
$R_1 = R_1 - 3R_2$ \\
\hline
$\left (
\begin{array}{rrr|r}
   0 &  1  &  -4 & -14\\
   0 &  0 &  0 & 0\\
   1  & 2 &  1 & 8\\
\end{array}  \right )$
&
$R_1 = R1 - 4R3$ \\
\hline
$\left (
\begin{array}{rrr|r}
   0 &  1  &  -4 & -14\\
   0 &  0 &  0 & 0\\
   1  & 0 &  9 & 36\\
\end{array}  \right )$
&
$R_3 = R_3 - 2R_1$ \\
\end{tabular} \\
The right side vector can be reconstruct using the first 2 columns. This shows that $\bar{x} \in cols(A)$, therefore, $\bar{x}$is the closet solution.\\
SVD Decomposition\\
$u =
\begin{pmatrix}
   -0.8478  &  0.4286  & -0.3123\\
   -0.4908  & -0.8571  &  0.1562\\
   -0.2008  &  0.2857  &  0.9370\\
\end{pmatrix}$\\

$s =
\begin{pmatrix}
    9.0554   &      0   &     0 \\
         0   & 5.7446   &      0\\
         0   &      0   & 0.0000\\
\end{pmatrix}$\\

$v =
\begin{pmatrix}
   -0.5051  &  0.0497  &  0.8616\\
   -0.8081  &  0.3233  & -0.4924\\
    0.3030  &  0.9450  &  0.1231\\
\end{pmatrix}$
\begin{equation*}
\bar{x} = V\frac{1}{\Sigma}U^Tb = 
\begin{pmatrix}
-12.3909 \\
10.2365 \\
9.5748
\end{pmatrix}
\end{equation*}
\section{Question 4}
\subsection*{4.a}
The matrix A describe an operation where any given vector will be projected onto the vector $u$ followed by a flattening to the lower dimension created by a vector that is perpendicular to $u$
\subsection*{4.b}
We can derive the eigenvalue using the $\lambda v = Av$ \\
$\lambda v = A(Av)$ \\
$\lambda v = A(\lambda v)$ \\
$\lambda v = \lambda (A v)$ \\
$\lambda v = \lambda^2 v$ \\
$v(\lambda^2 - \lambda) = 0$ \\
The only solutions that satisfies the equation $\lambda^2 - \lambda = 0$ are either 0 or 1. Therefore the eigenvalues must either be 0 or 1.

\subsection*{4.c}
The nullspace for A would be the .
\subsection*{4.d}
While the property of an idempotent matrix is AA = A. The following is a proof that any element in the $uu^T$ element will lead to itself.
\begin{equation}
Let\space uu^t = A =
\begin{pmatrix}
   {p_1}^2  &  p_1p_2  &  p_1p_3 & \dots & p_1p_n\\
    p_1p_2  &  {p_2}^2  &  p_2p_3 & \dots & p_2p_n\\
    \dots & \ddots & \ddots & \ddots & \dots \\
    p_1p_n &  p_np_2  &  p_np_3 & \dots & p_np_n\\
\end{pmatrix} \\
\end{equation}
The following is the multiplication for the first element.
\begin{equation}
AA(1,1) = {p_1}^2 + (p_1p_2)^2 + (p_1p_3)^2 + \dots + (p_1p_n)^2 = {p_1}^2 \sum_{i = 1}^{n}{{p_i}^2}
\end{equation}
since we know the length of the unit vector is $ = 1 = \sqrt{\sum_{i = 1}^{n}{{p_i}^2}} = 1 = \sum_{i = 1}^{n}{{p_i}^2}$. By subing it into the equations, we get
$ = {p_1}^2$ which is the same as the original element. Because of the symmetric property of matrix. All of the elements are follow similar construct and can be prove using the same method as above. Therefore $uu^t$ is an idempotent matrix. By the definition of idempotent, we know that subtraction between two idempotent matrix also leads to an idempotent matrix. Therefore the operation $I - uu^t$ will also create an idempotent matrix. Therefore $A$ would be an idempotent matrix and $A^2 = AA = A$.

\section{Question 5}
The solution was inspired by the work titled Least-Squares Fitting of Two 3-D Point Sets by Arun1988(link: http://162.105.204.96/teachers/yaoy/Fall2011/arun.pdf).\\
The aim of the algorthim is to minimize $\sigma^2 = \sum_{i = 1}^{N}||q_i - (Rp_i + T)||^2$\\
\lstinputlisting[style=Matlab-editor]{pointSolver.m}
\end{document}