\documentclass{article}
\usepackage{graphicx}
\usepackage{amsmath}
\usepackage[framed,numbered]{matlab-prettifier}

\begin{document}

\title{Math 16-811 - HW1 - resubmit}
\author{Xiang Zhi Tan}

\maketitle
\subsection*{2.b}
Let $ L = I, A' = A$ \\
\begin{tabular}{|l|c|p{5cm}|}
\hline
\textbf{L} & \textbf{A\'} & \textbf{Operations}\\
\hline
$\begin{pmatrix}
    1 & 0 & 0 & 0 & 0\\
    \frac{1}{2} & 1 & 0 & 0 & 0\\
    0 & 0 & 1 & 0 & 0\\
    0 & 0 & 0 & 1 & 0\\
    0 & 0 & 0 & 0 & 1\\
\end{pmatrix}$
&
$\begin{pmatrix}
    4 & 8 & 0 & 0 \\
    0 & -4 & -2 & 0 \\
    0 & 4 & -1 & 0 \\
    0 & -2 & 0 & 2 \\
    0 & 0 & 2 & -1 \\
\end{pmatrix}$
&
$R_2 = R_2 - \frac{1}{2} R_1 $ \\
\hline
$\begin{pmatrix}
    1 & 0 & 0 & 0 & 0\\
    \frac{1}{2} & 1 & 0 & 0 & 0\\
    0 & -1 & 1 & 0 & 0\\
    0 & \frac{1}{2} & 0 & 1  &0\\
    0 & 0 & 0 & 0 & 1\\
\end{pmatrix}$
&
$\begin{pmatrix}
    4 & 8 & 0 & 0 \\
    0 & -4 & -2 & 0 \\
    0 & 0 & -3 & 0 \\
    0 & 0 & 1 & 2 \\
    0 & 0 & 2 & -1 \\
\end{pmatrix}$
&
$R_3 = R_3 + R_2$ \newline
$R_4 = R_4 - \frac{1}{2} R_2$\\
\hline
$\begin{pmatrix}
    1 & 0 & 0 & 0 & 0\\
    \frac{1}{2} & 1 & 0 & 0  & 0\\
    0 & -1 & 1 & 0 & 0\\
    0 & \frac{1}{2} & -\frac{1}{3} & 1 & 1\\
    0 & 0 & 0 & 0 & 1\\
\end{pmatrix}$
&
$\begin{pmatrix}
    4 & 8 & 0 & 0 \\
    0 & -4 & -2 & 0 \\
    0 & 0 & -3 & 0 \\
    0 & 0 & 0 & 2 \\
    0 & 0 & 2 & -1 \\
\end{pmatrix}$
&
$R_4 = R_4 + \frac{1}{3} R_3$\\
\hline
$\begin{pmatrix}
    1 & 0 & 0 & 0 & 0\\
    \frac{1}{2} & 1 & 0 & 0 & 0 \\
    0 & -1 & 1 & 0 & 0\\
    0 & \frac{1}{2} & -\frac{1}{3} & 1 & 0\\
    0 & 0 & -\frac{2}{3} & -\frac{1}{2} & 1\\
\end{pmatrix}$
&
$\begin{pmatrix}
    4 & 8 & 0 & 0 \\
    0 & -4 & -2 & 0 \\
    0 & 0 & -3 & 0 \\
    0 & 0 & 0 & 2 \\
    0 & 0 & 0 & 0 \\
\end{pmatrix}$
&
$R_5 = R_5 + \frac{2}{3}R_3$ \newline
$R_5 = R_5 + \frac{1}{2}R_4$\\
\hline
\end{tabular} \\
Therefore: \\
\begin{equation*}
\begin{aligned}
P &=
\begin{pmatrix}
    1 & 0 & 0 & 0 & 0\\
    0 & 1 & 0 & 0 & 0\\
    0 & 0 & 1 & 0 & 0\\
    0 & 0 & 0 & 1 & 0\\
    0 & 0 & 0 & 0 & 1\\
\end{pmatrix} \\
L &= 
\begin{pmatrix}
    1 & 0 & 0 & 0 & 0\\
    \frac{1}{2} & 1 & 0 & 0 & 0 \\
    0 & -1 & 1 & 0 & 0\\
    0 & \frac{1}{2} & -\frac{1}{3} & 1 & 0\\
    0 & 0 & -\frac{2}{3} & -\frac{1}{2} & 1\\
\end{pmatrix} \\
D &= 
\begin{pmatrix}
    4 & 0 & 0 & 0 \\
    0 & -4 & 0 & 0 \\
    0 & 0 & -3 & 0 \\
    0 & 0 & 0 & 2 \\
\end{pmatrix} \\
U &= 
\begin{pmatrix}
    1 & 2 & 0 & 0 \\
    0 & 1 & \frac{1}{2} & 0 \\
    0 & 0 & 1 & 0 \\
    0 & 0 & 0 & 1 \\
\end{pmatrix} \\
\end{aligned}
\end{equation*}
SVD Decomposition \\
\begin{equation*}
\begin{aligned}
u = &
\begin{pmatrix}
   -0.9005  & -0.0155  &  0.3562 &  -0.1617 &  -0.1895 \\
   -0.0891  & -0.7311  &  0.1847  &  0.5290 &   0.3789 \\
   -0.3787  &  0.0340  & -0.7352 &  -0.1479  &  0.5413 \\
    0.1942  & -0.3642  &  0.2836 &  -0.8023  &  0.3248 \\
    0.0078  &  0.5757  &  0.4670  &  0.1686  & 0.6496 \\
\end{pmatrix}
\\
s =&
\begin{pmatrix}
    9.8844    &     0   &      0     &    0\\
         0 &  3.3506   &      0     &    0\\
         0    &     0  &  2.3134    &     0\\
         0    &     0    &     0   & 1.9288\\
         0    &     0     &    0    &     0\\
\end{pmatrix}
\\
v =&
\begin{pmatrix}
   -0.3824  & -0.4549  &  0.7755  &  0.2132 \\
   -0.9214  &  0.2210  & -0.2847  & -0.1456 \\
    0.0579  &  0.7699  &  0.5619  & -0.2970 \\
    0.0385  & -0.3892  &  0.0434  & -0.9193 \\
 \end{pmatrix}
 \end{aligned}
 \end{equation*}



\section{Question 5}
I received help from Devin Schwab for this question. The solution was based on the work by Arun(1988) titled Least-Squares Fitting of Two 3-D Point Sets(link: http://162.105.204.96/teachers/yaoy/Fall2011/arun.pdf).\\

This problem asked to find the rotation and translation function that transforms $p_i$ to $q_i$ for where$i = 1 ... n$. The equation of function can be written out as following:
\begin{equation*}
q_i = Rp_i + t
\end{equation*}
Where $R$ is the rotation matrix and $t$ is the translation vector.\\
Since we assume that our measurement of $q_i$ contains error or noise, we are trying to find the rotation matrix and translation vector with the least error. The error function of the equation is as following:
\begin{equation*}
E = \sum_{i = 1}^{N}||q_i - (Rp_i + T)||^2
\end{equation*}
Each point are first rotated than moved by the translation matrix. Because rotation are done with a fixed centroid, the centroid of all the points should be the same and did not change by the rotation. Therefore, we could calculate the centroid for both $p_i$ and $q_i$ and move them to the origin point on the plan. This will make our new points only be affected by rotation instead of both rotation and translation. Here's the equation for the operations. First, we calculate the centroid\\
\begin{equation*}
\begin{aligned}
p_c &= \frac{1}{n}\sum_{i=1}^{n}p\\
q_c &= \frac{1}{n}\sum_{i=1}^{n}q
\end{aligned}
\end{equation*}
we then subtract the centroid from each points to create a set of points:
\begin{equation*}
\begin{aligned}
p'_i =& p_i - p_c\\
q'_i =& q_i - q_c
\end{aligned}
\end{equation*}
Our error function is now updated to the following:
\begin{equation*}
E = \sum_{i = 1}^{N}||(q'_i - Rp'_i)||^2
\end{equation*}
By the theorem in which $u \dot u = ||u||^2$ where $||u||$is the length of the vector and identity of $AB = (B^TA^T)^T$. We can rewrite the equation as:
\begin{equation*}
\begin{aligned}
E =& \sum_{i = 1}^{N}(q'_i - Rp'_i) \bullet (q'_i - Rp'_i)\\
=& \sum_{i = 1}^{N}(q'_i - Rp'_i)^{T}(q'_i - Rp'_i)\\
=& \sum_{i = 1}^{N}(q'_i)^T(q'_i) - (q'_i)^T(Rp'_i) - (Rp'_i)^T(q'_i) + (Rp'_i)^T(Rp'_i)\\
=& \sum_{i = 1}^{N}(q'_i)^T(q'_i) - (q'_i)^T(Rp'_i) - (q'_i)^T(Rq'_i) + (R)^T(p'_i)^T(Rp'_i)\\
=& \sum_{i = 1}^{N}(q'_i)^T(q'_i) + (p'_i)^T(p'_i) - 2(q'_i)^T(Rp'_i)
\end{aligned}
\end{equation*}
Because we are trying to minimize the error, the only term that is important would be the negative term of $2(q'_i)^T(Rp'_i)$. where we hope to maximize it. This leads to this equation
\begin{equation*}
F = \sum_{i = 1}^{N} (q'_i)^T(Rp'_i)
\end{equation*}
Because $(q'_i)^T$ is a 1 x 3 vector and $RP'_i$ is a 3 x 1 vector. F is the inner product of  $(q'_i)^T$ and $RP'_i$. Because the inner product is the trace of the outer product of the 2 vectors. The equation is rewritten as:
\begin{equation*}
F = Trace(\sum_{i = 1}^{N} Rp'_i{q'_i}^T)
\end{equation*}
We will define a 3x3 matrix called $H$ that is $\sum_{i = 1}^{N} p'_i {q'i}^{T}$. This will lead the equation to be:
\begin{equation*}
F = Trace(RH)
\end{equation*}
We can decompose $H$ using SVD into $U\Sigma V^T$. Let $R = BX$ and $VU^T$. Therefore, $XH = VU^TU\Sigma V^T$ which is equal to $V\Sigma V^T$. Because $B$ is orthonormal(Rotation matrix are orthonormal), the following lemma can be use:\\
\textbf{Lemma:} For any positive definite matrix $AA^T$ and any orthonormal matrix $B$,
\begin{equation*}
  Trace(AA^T) \geq Trace(BAA^T)
\end{equation*}
Which means $Trace(XH) \geq Trace(BXH)$. $H$ is fixed because it is generated from the points given, therefore only by maximizing $X$ can we maximize $F$.From the previous equation and substituting $H=U\Sigma V^T$, we can find $R$ as $R = X = VU^T$\\
After finding $R$, we can find the translation matrix by applying rotation to the centroids and find the difference between then.
\begin{equation*}
T = p' - Rp
\end{equation*}
\lstinputlisting[style=Matlab-editor]{pointSolver.m}
\end{document}